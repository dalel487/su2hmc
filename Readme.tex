In this folder there are two directories:
start_firstconf ,
production .

The "start_firstconf" can be used to generate just the first configuration
(random start) that can be then used in "production" to thermalise and 
then produce the configurations.

To compile you need the command "make" which, on BG/Q Edimburgh, has to be done
on the machine called "bgqfe4". After compilation you can submit on the 
machine "bgqfe2". Our working directory is  /dirac1/work/dp006/dp006/"yourusername" .

To sumbit the jobs you have to use the command: 
"qsub su2start.sh" (in start_firstconf) and
"qsub su2production.sh" (in production).
To check the jobs the command "qstat", to kill it "qdel".

The file "sizes.h" and the scripts to submit the jobs (su2start.sh and su2production.sh) 
have been set to run on 2048 cores, using all 16 cores per node and 4 threads. 
Of course you can modify them but I checked this is a good configuration. 
Note that on 16^3x32 this means having a subvolume per core of 2^3x8.

In the directory "production", in the script su2production.sh, you have to modify 
and/or create, your REPOSITORY directory;
I fixed it to:
REPOSITORY=/dirac1/work/dp006/dp006/dc-boz1/SU2/Configs/V${ss}x${tt}/J${jj}/MU$m
but of course you can modify it.

The only file you should modify is the script to submit the jobs where the parameters
of the theory and of the simulation are, i.e. su2start.sh and su2production.sh .

In particular:
ss=16
tt=32
beta=1.7
kappa=0.1810
mu=0.15
j=0.04

and

dt=0.0032
iterlen=312

To check if these last two are ok, there is the script 
extract/extract-info-acceptance.sh
which gives you the acceptance ratio of the hmc algorithm.

Note that if you change the volume you have to change it ALSO in the file size.h.

The other parameter you can change in the script su2start.sh is "nconf", 
i.e. the total number of configurations generated in the first run.
Now it is "nconf=5", and because it is only for the first one, it is ok.

In the script su2production.sh instead, there is again "nconf" 
(you can increase its value according to the fact it can run for a 
maximum of 24 hours consecutively), there is the total number of
configuration you want to geenrate "end=1000", and the parameter
"save=5" which say you that the configuration are saved every 5
trajectory (this parameter is present also in the code su2hmc.f, so if
you modify it you have to change it in both places).

Pietro Giudice,
06/sept/2013.


